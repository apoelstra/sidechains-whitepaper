% Default course lecture note template by asp 
\documentclass[letterpaper]{article}
\usepackage[utf8]{inputenc}
\usepackage[T1]{fontenc}
\usepackage[english]{babel}
\usepackage[top=3cm, bottom=3cm, left=3.85cm, right=3.85cm]{geometry}
\usepackage[onehalfspacing]{setspace}
\usepackage{amsmath,amssymb,wasysym,amsthm}
\usepackage{graphicx}
\usepackage{mathptmx}
\usepackage{tikz}
\usetikzlibrary{positioning}
\usepackage{hyperref}
\usepackage{gitinfo}
\usepackage{xspace}
\usepackage{lineno}
\usepackage{algorithm}
\usepackage{algorithmicx}
\usepackage{algpseudocode}
\usepackage{verbatimbox}
\usepackage{multirow}
\usepackage{scalerel}
\usepackage[tablename=Figure]{caption}
\usepackage[title]{appendix}
\renewcommand{\algorithmicrequire}{\textbf{Input:}}
\renewcommand{\algorithmicensure}{\textbf{Output:}}
\title{Enabling~Blockchain~Innovations~with~Pegged~Sidechains}
\author{
  Adam Back,
  Matt Corallo,
  Luke Dashjr,\\
  Mark Friedenbach,
  Gregory Maxwell,\\
  Andrew Miller,
  Andrew Poelstra,\\
  Jorge Tim\'on,
  and
  Pieter Wuille\footnote{
    \texttt{adam@cypherspace.org},
    \texttt{mattc@bluematt.me},
    \texttt{luke@dashjr.org},
    \texttt{mark@friedenbach.org},\newline
    \texttt{greg@xiph.org},
    \texttt{amiller@cs.umd.edu},
    \texttt{apoelstra@wpsoftware.net},
    \texttt{jtimon@jtimon.cc},\newline
    \texttt{pieter.wuille@gmail.com}}
\footnote{This work was partially supported by Blockstream, a company founded by several of the authors. Many of the concepts discussed in this paper were developed before
Blockstream existed; sidechain technology is an open proposal by the authors
themselves.}
}
\date{\gitAuthorDate{} (commit \texttt{\gitAbbrevHash)}}


% my usual newcommands
\newcommand{\pk}{\textnormal{\texttt{pk}}}
\newcommand{\sk}{\textnormal{\texttt{sk}}}

% terms that might change
\newcommand{\mainchain}{parent chain\xspace}
\newcommand{\mainchains}{parent chains\xspace}
\newcommand{\maincoin}{parent chain coin\xspace}
\newcommand{\maincoins}{parent chain coins\xspace}
\newcommand{\maintx}{transaction on the parent chain\xspace}
\newcommand{\maintxes}{transactions on the parent chain\xspace}
\newcommand{\sidechain}{sidechain\xspace}
\newcommand{\sidechains}{sidechains\xspace}
\newcommand{\sidecoins}{sidechain coins\xspace}

\newcommand{\peggedsidechain}{pegged sidechain\xspace}
\newcommand{\peggedsidechains}{pegged sidechains\xspace}
\newcommand{\altcoin}{altcoin\xspace}
\newcommand{\altcoins}{altcoins\xspace}
\newcommand{\altchain}{altchain\xspace}
\newcommand{\altchains}{altchains\xspace}

\newcommand{\hashsig}{DMMS\xspace}
\newcommand{\hashsigs}{DMMSes\xspace}
\newcommand{\consensus}{consensus system\xspace}

\newcommand{\nalts}{hundreds\xspace}

\newcommand{\onewp}{one-way peg\xspace}
\newcommand{\pegtarget}{target\xspace}
\newcommand{\onewpped}{one-way pegged\xspace}
\newcommand{\twowp}{two-way peg\xspace}
\newcommand{\twowpped}{two-way pegged\xspace}

\newcommand{\oracle}{functionary\xspace}
\newcommand{\oracles}{functionaries\xspace}

%diagram
\newcommand{\extline}{$\scriptsize$-$\normalsize$\!}
\newcommand{\lextlineend}{$\scriptsize$\lhd\!$\normalsize$}
\newcommand{\rextlineend}{$\scriptsize\rule{.1ex}{0ex}$\rhd$\normalsize$}

\newcounter{index}
\newcommand\extlines[1]{%
  \setcounter{index}{0}%
  \whiledo {\value{index}< #1}
  {\addtocounter{index}{1}\extline}
}

\newcommand\rextlinearrow[2]{$
  \setbox0\hbox{$\extlines{#2}\rextlineend$}%
  \tiny$%
  \!\!\!\!\begin{array}{c}%
  \textnormal{#1}\\%
  \usebox0%
  \end{array}%
  $\normalsize$\!\!%
}

\newcommand\lextlinearrow[2]{$
  \setbox0\hbox{$\lextlineend\extlines{#2}$}%
  \tiny%
  $%
  \!\!\!\!\begin{array}{c}%
  \textnormal{#1}\\%
  \usebox0%
  \end{array}%
  $\normalsize$\!\!%
}

\renewcommand\lextlinearrow[2]{%
  \setbox0\hbox{$\lextlineend\extlines{#2}$}%
  \shortstack{$\textnormal{#1}$\\\addvbuffer[-.7ex -.3ex]{\usebox0}}%
}

\renewcommand\rextlinearrow[2]{%
  \setbox0\hbox{$\extlines{#2}\rextlineend$}%
  \shortstack{$\textnormal{#1}$\\\addvbuffer[-.7ex -.3ex]{\usebox0}}%
}


\begin{document}

\maketitle
\hyphenpenalty=15000
\pretolerance=10000
\interfootnotelinepenalty=10000

\begin{abstract}
Since the introduction of Bitcoin\cite{nakamoto2009} in 2009, and the multiple computer
science and electronic cash innovations it brought, there has been
great interest in the potential of decentralised cryptocurrencies. At the
same time, implementation changes to the consensus-critical parts of Bitcoin
must necessarily be handled very conservatively. As a result, Bitcoin has greater
difficulty than other Internet protocols in adapting to new demands and accommodating new innovation.

We propose a new technology, \emph{\peggedsidechains}, which enables bitcoins and
other ledger assets to be transferred between multiple blockchains. This gives
users access to new and innovative cryptocurrency systems using the assets
they already own. By reusing Bitcoin's currency, these systems can more easily
interoperate with each other and with Bitcoin, avoiding the liquidity shortages
and market fluctuations associated with new currencies. Since \sidechains
are separate systems, technical and economic innovation is not hindered. Despite
bidirectional transferability between Bitcoin and \peggedsidechains, they are
isolated: in the case of a cryptographic break (or malicious design) in a sidechain,
the damage is entirely confined to the \sidechain itself.

This paper lays out \peggedsidechains, their implementation requirements, and
the work needed to fully benefit from the future of interconnected blockchains.
\end{abstract}

\clearpage
\tableofcontents

~\\~\\~\\~\\~\\~\\~\\
\paragraph{License.} This work is released into the public domain.
\clearpage
\modulolinenumbers[10]
\linenumbers

\section{Introduction}


David Chaum introduced digital cash as a research topic in 1983, in a setting with
a central server that is trusted to prevent \emph{double-spending}\cite{chaum1983}.
To mitigate the privacy risk to individuals from this central trusted party, and
to enforce fungibility, Chaum introduced the  \emph{blind signature}, which he
used to provide a cryptographic means to prevent linking of the central server's
signatures (which represent coins), while still allowing the central server to
perform double-spend prevention. The requirement for a central server became
the Achilles' heel of digital cash\cite{grigg1999}. While it is possible to
distribute this single point of failure by replacing the central server's
signature with a \emph{threshold signature} of several signers, it is important
for auditability that the signers be distinct and identifiable. This still
leaves the system vulnerable to failure, since each signer can fail, or be
made to fail, one by one.

In January of 2009, Satoshi Nakamoto released the first widely used implementation
of peer-to-peer trustless electronic cash\cite{nakamoto2009}, replacing the central
server's signature with a consensus mechanism based on proof of work\cite{back2002}, with economic
incentives to act cooperatively. Bitcoin tracks payments by aggregating them into
\emph{blocks}, each with an associated \emph{blockheader}, which cryptographically
commits\footnote{A \emph{commitment} is a cryptographic object which is computed
from some secret data, but does not reveal it, such that the data cannot be changed
after the fact. An example of a commitment is a \emph{hash}: given data $x$, one can
publish $H(x)$ where $H$ is a hash function, and only later reveal $x$. Verifiers
can then confirm that the revealed value is the same as the original value by
computing $H(x)$ themselves.} to: the contents of the block, a timestamp, and
the previous blockheader. The commitments to previous headers form a
\emph{blockchain}, or \emph{chain}, which provides a well-defined ordering for transactions.

We observe that Bitcoin's blockheaders can be regarded as an example of a \emph{dynamic-membership
multi-party signature} (or \emph{\hashsig}), which we consider to be of independent
interest as a new type of group signature. Bitcoin provides the first embodiment of
such a  signature, although this has not appeared in the literature until now.  A
\hashsig is a digital  signature formed by a set of signers which has no fixed size.
Bitcoin's blockheaders are \hashsigs because their proof-of-work has the property
that anyone can contribute with no enrolment process. Further, contribution is
weighted by computational power rather than one  threshold signature contribution
per party,  which allows anonymous membership without risk of a \emph{Sybil attack}
(when one party joins many times and has disproportionate input into the signature).
For this reason, the \hashsig has also been described as a solution to the Byzantine
Generals Problem\cite{aspnes+jackson+krishnamurthy2005}.

Because the blocks are chained together, Bitcoin's \hashsig is cumulative: any
chain (or chain fragment) of blockheaders is also a \hashsig on its first block,
with computational strength equal to the sum of the strengths of the
\hashsigs it is composed of. Nakamoto's key innovation is the aforementioned use
of a \hashsig as a signature of \emph{computational power} rather than a
signature of \emph{knowledge}. Because signers prove computational work, rather
than proving secret knowledge as is typical for digital signatures, we refer to them
as \emph{miners}. To achieve stable consensus on the blockchain history, economic
incentives are provided where miners are rewarded with fees and subsidies in
the form of coins that are valuable only if the miners form a shared valid
history, incentivising them to behave honestly. Because the strength of Bitcoin's
cumulative \hashsig is directly proportional to the total computational power
contributed by all miners\cite{poelstra2014-3}, it
becomes infeasible for a computational minority to change the
chain. If they try to revise the \hashsig-secured ledger, they will fall
behind and be continually unable to catch up to the moving target of the
progressing consensus blockchain.

Because the miners do not form an identifiable set, they cannot have discretion over the
rules determining transaction validity. Therefore, Bitcoin's rules must be determined
at the start of its history, and new valid transaction forms cannot be added except with the
agreement of every network
participant. Even with such an agreement, changes are difficult to deploy because they
require all participants to implement and execute the new rules in exactly the same way,
including edge cases and unexpected interactions with other features.

For this reason, Bitcoin's objective is relatively simple: it is a blockchain
supporting the transfer of a single native digital asset, which is not redeemable
for anything else. This allowed
many simplifications in the implementation, but real-world demands are now challenging
those simplifications. In particular, current innovation is focused
around the following areas:
\begin{enumerate}
\item There are trade-offs between scalability and decentralisation. For example,
a larger block size would allow the network to support a higher transaction rate, at
the cost of placing more work on validators --- a centralisation risk.

Similarly, there are trade-offs between security and cost. Bitcoin
stores every transaction in its history with the same level of irreversibility.
This is expensive to maintain and may not be appropriate for low value
or low-risk transactions (\emph{e.g.} where all parties already have shared
legal infrastructure in place to handle fraud).

These trade-offs should be made for each
transaction, as transactions vary widely in value and risk profile. However, Bitcoin by
construction supports only a ``one size fits all'' solution.
\item There are many more trade-offs for blockchain features. For example, Bitcoin's
script could be more powerful to enable succinct and useful contracts, or could be
made less powerful to assist in auditability.

\item There are assets besides currencies that may be traded on
blockchains, such as IOUs and other contracts, as well as smart property \cite{szabo1997}.

\item There is a risk of monoculture: Bitcoin is composed of many cryptographic
components, any one of whose failures could cause a total loss of value. If
possible, it would be prudent not to secure every bitcoin with the same set
of algorithms.

\item New technology might enable new features not imagined when Bitcoin was
first developed. For example, privacy and censorship-resistance could be improved by
use of cryptographic accumulators\cite{mouton2013}, ring signatures\cite{vansaberhagen2013},
or Chaumian blinding\cite{chaum1983}.
\item Even if there is a pressing need to do so, there is no safe upgrade path for
Bitcoin, in the sense that all participants must act in concert for any change
to be effected. There is consensus amongst Bitcoin developers that changes to Bitcoin
must be done slowly, cautiously, and only with clear assent from the community.

The fact that functionality must be broadly acceptable to gain adoption limits participants'
personal freedom and autonomy over their own coins. Small groups are unable to implement features,
such as special-purpose script extensions\cite{jl20122013}, because they lack
broad consensus.

\end{enumerate}

An early solution to these problems with Bitcoin has been the development of alternate
blockchains, or \emph{\altchains}, which share the Bitcoin codebase
except for modifications to address the above concerns. However,
implementing technical changes through the creation of independent but essentially
similar systems is problematic.

One problem is infrastructure fragmentation:
because each \altchain uses its own technology stack, effort is frequently duplicated and lost.
Because of this, and because implementers of \altchains may fail to clear the
very high barrier of security-specific domain knowledge in Bitcoin\cite{poelstra2014-1},
security problems are often duplicated across \altchains while their fixes are
not. Substantial resources must be spent finding or building the expertise to
review novel distributed cryptosystems, but when they are not, security weaknesses
are often invisible until they are exploited. As a result, we have seen a volatile, unnavigable environment develop, where the most visible projects
may be the least technically sound.
As an analogy, imagine an Internet where every website used its own TCP
implementation, advertising its customised checksum and packet-splicing
algorithms to end users. This would not be a viable environment, and neither is
the current environment of \altchains.

A second problem is that such \altchains, like Bitcoin, typically have their own native cryptocurrency, or \emph{\altcoin}, with a floating price. To access the \altchain, users must use a market to obtain this currency, exposing them to the
high risk and volatility associated with new currencies.
Further, the requirement to independently solve the problems of
initial distribution and
valuation, while at the same time contending with adverse network effects and
a crowded market, discourages technical innovation while at the same time encouraging market
games.
This is dangerous not only to those directly participating in these systems,
but also to the cryptocurrency industry as a whole. If the field is seen as
too risky by the public, adoption may be hampered, or 
cryptocurrencies might be deserted entirely (voluntarily or legislatively).

It appears that we desire a world in which interoperable \altchains can be easily
created and used, but without unnecessarily fragmenting markets and development.
In this paper, we argue that it is possible to simultaneously achieve these
seemingly contradictory goals.
The core observation is that ``Bitcoin'' the blockchain is conceptually
independent from ``bitcoin'' the asset: if we had technology to support
the movement of assets between blockchains, new
systems could be developed which users could adopt by simply reusing the
existing bitcoin currency\footnote{We use bitcoin as an example because its strong network effects
make it likely that users will prefer it over other, newer assets. However, any \altcoin can be
adapted to be usable with \sidechains.}.

We refer to such interoperable blockchains as \emph{\peggedsidechains}. We
will give a precise definition in Section \ref{def}, but for now we
list the following desired properties for \peggedsidechains:
\begin{enumerate}
\item Assets which are moved between sidechains should be
able to be moved back by whomever their current holder is, and nobody else (including
previous holders).

\item Assets should be moved without counterparty risk; that is, there should be no ability for a
dishonest party to prevent the transfer occurring.

\item Transfers should be atomic, \emph{i.e.} happen entirely or not at
all. There should not be failure modes that result in loss or allow fraudulent
creation of assets.

\item Sidechains should be \emph{firewalled}: a bug in one \sidechain enabling
creation (or theft) of assets in that chain should not result in creation or theft of assets
on any other chain.
\item Blockchain reorganisations\footnote{A \emph{reorganisation}, or \emph{reorg},
occurs locally in clients when a previously accepted chain is overtaken by a competitor
chain with more proof of work, causing any blocks on the losing side of the fork to be
removed from consensus history.} should be handled cleanly, even during transfers;
any disruption should be localised to the sidechain on which it occurs. In general,
\sidechains should ideally be \emph{fully independent}, with
users providing any necessary data from other chains. Validators of a \sidechain should only be required to track another chain if that is an explicit consensus rule of the \sidechain itself.

\item Users should not be required to track \sidechains that they are not
actively using.
\end{enumerate}

An early solution was to ``transfer'' coins by destroying bitcoins in a publicly
recognisable way\footnote{This is also known as a \emph{one-way peg}, to contrast
with \emph{two-way peg}, which we introduce later on.}, which would be detected by a new blockchain to allow creation
of new coins\cite{back2013-1}. This is a partial solution to the problems listed
above, but since it allows only unidirectional transfers between chains, it is not sufficient
for our purposes.

Our proposed solution is to transfer assets by providing proofs of possession in the
transferring transactions themselves, avoiding the need for nodes to
track the sending chain. On a high level, when moving assets from one
blockchain to another, we create a transaction on the first blockchain
locking the assets, then create a transaction on the second blockchain
whose inputs contain a cryptographic proof that the lock was done correctly.
These inputs are tagged with an asset type, \emph{e.g.} the genesis hash of
its originating blockchain.

We refer to the first blockchain as the \emph{\mainchain}, and the second simply as
the \emph{\sidechain}. In some models, both chains are treated symmetrically,
so this terminology should be considered relative. Conceptually, we would like to
transfer an asset from the (original) \mainchain to a \sidechain, possibly onward to another
\sidechain, and eventually back to the \mainchain, preserving the original asset.
Generally we can think of the \mainchain as being Bitcoin and the \sidechain
as one of many other blockchains. Of course, \sidecoins
could be transferred between \sidechains, not just to and from Bitcoin;
however, since any coin originally moved from Bitcoin could be moved back,
it would nonetheless remain a bitcoin.

This lets us solve the problem of fragmentation described in the previous section, which
is good news for cryptocurrency developers who want to focus solely
on technical innovation.

Furthermore, because \sidechains transfer existing assets from the \mainchain rather than creating new ones, \sidechains cannot cause unauthorised creation
of coins, relying instead on the \mainchain to maintain the security and scarcity of its
assets\footnote{Of course, \sidechains are able to support their own assets, which they
would be responsible for maintaining the scarcity of. We mean to emphasise that they can only affect the scarcity of themselves and their child chains.}.

Further still, participants do not need to be as concerned that their holdings
are locked in a single experimental \altchain, since \sidecoins can be
redeemed for an equal number of \mainchain coins. This provides an exit
strategy, reducing harm from unmaintained software.

On the other hand, because \sidechains are still blockchains independent of
Bitcoin, they are free to experiment with new transaction designs, trust
models, economic models, asset issuance semantics, or cryptographic features.
We will explore many of the possibilities for \sidechains further in Section
\ref{app}.

An additional benefit to this infrastructure is that making changes to Bitcoin itself
becomes much less pressing: rather than orchestrating a fork which all parties
need to agree on and implement in tandem, a new ``changed Bitcoin''
could be created as a \sidechain. If, in the medium term, there were wide agreement
that the new system was an improvement, it may end up seeing significantly more use than Bitcoin. As there are no
changes to \mainchain consensus rules, everyone can switch in their own time without
any of the risks associated with consensus failure. Then, in the longer term,
the success of the changes in the \sidechain would provide the needed confidence
to change the \mainchain, if and when it is deemed necessary to do so.

\section{Design rationale}

\emph{Trustlessness} is the property of not relying on
trusting external parties for correct operation, typically by enabling all
parties to verify on their own that information is correct. For example,
in cryptographic signature systems trustlessness is an implicit requirement
(signature systems where an attacker can forge signatures would be
considered utterly broken). While this is not typical for distributed
systems, Bitcoin does provide trustless operation for most parts of its system\footnote{This
is true for almost all aspects of Bitcoin: a user running a full node will
never accept a transaction that is directly or indirectly the result of
counterfeiting or spending without proving possession. However, trustless
operation is not possible for preventing double spending, as there is no
way to distinguish between two conflicting but otherwise valid transactions.
Instead of relying on a centralised trusted party or parties to take on
this arbitration function like Bitcoin's predecessors, Bitcoin reduces
the trust required --- but does not remove it --- by using a \hashsig
and economic incentives.}.

A major design goal of \peggedsidechains is to minimise additional trust over Bitcoin's model. The hard part is securing transfers of coins between \sidechains: the receiving chain must see that the coins on the sending chain were correctly locked. Following Bitcoin's lead, we propose solving this using \hashsigs. Although it is possible to use a simple trust-based solution involving fixed signers (see Appendix \ref{padaptor}) to verify locking of coins, there are important reasons to avoid the introduction of single points of failure:

\begin{itemize}
\item Trusting individual signers does not only mean expecting them to behave honestly;
they must also never be compromised, never leak secret key material, never be coerced,
and never stop participating in the network.

\item Because digital assets are long-lived, any trust requirements must be as well.
Experience has shown that trust requirements are
dangerous expectations even for timespans on the order of months, let alone the
generational timespans we expect financial systems to last.

\item Digital currencies were unable to gain traction until Bitcoin was able to eliminate
single points of failure, and the community is strongly averse to the introduction of such
weaknesses. Community mistrust is reinforced by financial events since 2007; public
trust in the financial system and other public institutions is likewise at historical lows.
\end{itemize}

\section{Two-way peg\label{def}}

The technical underpinning of \peggedsidechains is called
the \emph{\twowp}. In this section we explain the workings thereof, beginning
with some definitions.

\subsection{Definitions}

\begin{itemize}
\item A \emph{coin}, or \emph{asset}, is a digital property whose controller
can be cryptographically ascertained.

\item A \emph{block} is a collection of transactions describing changes in
asset control.

\item A \emph{blockchain} is a well-ordered collection of blocks, on which
all users must (eventually) come to consensus. This determines the history
of asset control and provides a computationally unforgeable time ordering
for transactions.

\item A \emph{reorganisation}, or \emph{reorg}, occurs locally in clients when a previously accepted chain is overtaken by a competitor chain with more proof of work, causing any blocks on the losing side of the fork to be removed from consensus history.

\item A \emph{\sidechain} is a blockchain that validates data from
other blockchains.

\item \emph{Two-way peg} refers to the mechanism by which coins are transferred
between \sidechains and back at a fixed or otherwise deterministic exchange rate.

\item A \emph{\peggedsidechain} is a \sidechain whose assets can be imported from
and returned to other chains; that is, a \sidechain that supports two-way pegged assets.

\item A \emph{simplified payment verification proof} (or \emph{SPV proof}\footnote{Named after the section `Simplified Payment Verification' in \cite{nakamoto2009}}) is
a \hashsig that an action occurred on a Bitcoin-like proof-of-work blockchain.

Essentially, an SPV proof is composed of (a) a list of blockheaders
demonstrating proof-of-work, and (b) a cryptographic proof that an
output was created in one of the blocks in the list.
This allows verifiers to check that some amount of work has been
committed to the existence of an output. Such a proof may be invalidated
by another proof demonstrating the existence of a chain with more work
which does not include the block which created the output.

Using SPV proofs to determine history, implicitly trusting that the longest
blockchain is also the longest correct blockchain, is done by so-called
\emph{SPV clients} in Bitcoin.
Only a dishonest collusion with greater than 50\% of the hashpower can persistently
fool an SPV client (unless the client is under a long-term Sybil attack, preventing it from seeing the actual longest chain), since the honest hashpower will not contribute work to an invalid chain.

Optionally, by requiring each blockheader to commit
to the blockchain's unspent output set\footnote{In Bitcoin, only the set
of \emph{unspent transaction
outputs (UTXO's)} is needed to determine the status of all coins. By constructing a
Merkle tree\cite{merkle1988}, we can commit to every element of the UTXO set using only
a single hash, minimising the blockheader space used.}, anyone in possession of an SPV
proof can determine the
state of the chain without needing to ``replay'' every block. (In Bitcoin, full
verifiers need to do this when they first start tracking the blockchain.)

As we will discuss in Appendix \ref{compactspv}, by including some additional
data in Bitcoin's block structure, we can produce smaller
proofs than a full list of headers, which will improve scalability.
Still, these proofs will be much larger than ordinary Bitcoin transactions. Fortunately,
they are not necessary for most transfers: holders of coins on each chain may exchange
them directly using atomic swaps\cite{nolan2013}, as described in Appendix \ref{atomicswaps}.
\end{itemize}

\subsection{Symmetric two-way peg\label{symmpeg}}

We can use these ideas to \emph{SPV peg} one \sidechain to another.
This works as follows: to transfer \maincoins into
\sidecoins, the \maincoins are sent to a special output on the \mainchain
that can only be unlocked by an \emph{SPV proof} of possession on the
\sidechain. To synchronise the two chains, we need to define two waiting periods:
\begin{enumerate}
\item The \emph{confirmation period} of a transfer between sidechains is a
duration for which a coin must be locked on the \mainchain before it can
be transferred to the \sidechain. The purpose of this confirmation period is
to allow for sufficient work to be created such that a denial of service attack in the next waiting period becomes more difficult. A typical confirmation period would be on the order of a
day or two.

After creating the special output on the \mainchain, the user waits out the confirmation period,
then creates a transaction on the \sidechain referencing this output,
providing an SPV proof that it was created and buried under sufficient
work on the the \mainchain.

The confirmation period is a per-sidechain security parameter, which trades
cross-chain transfer speed for security.

\item The user must then wait for the \emph{contest period}. This is a
duration in which a newly-transferred coin may not be spent on the \sidechain.
The purpose of a contest period is to prevent
double-spending by transferring previously-locked coins during a reorganisation.
If at any point during this delay, a new proof is
published containing a chain with more aggregate work which does not include the
block in which the lock output was created, the
conversion is retroactively invalidated. We call this a \emph{reorganisation
proof}.

All users of the \sidechain have an incentive to produce reorganisation proofs
if possible, as the consequence of a bad proof being admitted is a dilution
in the value of all coins.

A
typical contest period would also be on the order of a day or two. To avoid these
delays, users will likely use atomic swaps (described in Appendix \ref{atomicswaps})
for most transfers, as long as a liquid market is available.
\end{enumerate}


While locked on the \mainchain, the coin can be freely transferred within
the \sidechain without further interaction with the \mainchain. However,
it retains its identity as a \maincoin, and can only be transferred back
to the same chain that it came from.

When a user wants to transfer coins from the \sidechain back to the
\mainchain, they do the same thing as the original transfer: send the
coins on the \sidechain to an SPV-locked output, produce a sufficient
SPV proof that this was done, and use the proof to unlock a number of
previously-locked outputs with equal denomination on the \mainchain.
The entire transfer process is demonstrated in Figure \ref{diagramtable}.

\begin{table}
\begin{tabular}{rlc}
\multicolumn{1}{c}{\large Parent Chain} & & \multicolumn{1}{c}{\large Sidechain}\\
\multicolumn{1}{c}{\vdots} &  & \multicolumn{1}{c}{\vdots}\\
Send to SPV-locked output && \\
Wait out confirmation period && \\
& \rextlinearrow{SPV Proof}{26} &\\
  && Wait out contest period \\
  && \multicolumn{1}{c}{\vdots} \\
  && (Intra-chain transfers) \\
  && \multicolumn{1}{c}{\vdots} \\
  && Send to SPV-locked output \\
  && Wait out confirmation period \\
& \lextlinearrow{SPV proof}{26} &\\
Contest period begins && \\
& \lextlinearrow{SPV reorganisation Proof}{26} &\\
Contest period ends (failed) && \\
& \lextlinearrow{New SPV proof}{26} &\\
Wait out contest period && \\
\multicolumn{1}{c}{\vdots} && \\
(Intra-chain transfers) && \\
\multicolumn{1}{c}{\vdots} &  & \multicolumn{1}{c}{\vdots}
\end{tabular}
\caption{Example two-way peg protocol.\label{diagramtable}}
\end{table}

Since \peggedsidechains may carry assets from many chains, and cannot
make assumptions about the security of these chains, it is important
that different assets are not interchangeable (except by an explicit
trade). Otherwise a malicious user may execute a theft by creating a
worthless chain with a worthless asset, move such an asset to a
\sidechain, and exchange it for something else. To combat this, sidechains
must effectively treat assets from separate \mainchains as separate asset types.

In summary, we propose to make the \mainchain and \sidechains do SPV validation of
data on each other. Since the \mainchain clients cannot be expected
to observe every \sidechain, users import proofs of work from the \sidechain
into the \mainchain in order to prove possession. In a symmetric two-way peg,
the converse is also true.

To use Bitcoin as the \mainchain, an extension to script which can recognise and validate such SPV proofs
would be required. At the very least, such proofs would need to be made compact enough to fit in a Bitcoin transaction. 
However, this is just a soft-forking change\footnote{A soft-forking change is
a change which only imposes further restrictions on what is valid inside the chain. See Section~\ref{softfork} for more information.},
without effect on transactions which do not use the new features.

\subsection{Asymmetric two-way peg}

The previous section was titled ``\emph{Symmetric} Two-Way Peg'' because
the transfer mechanisms from \mainchain to \sidechain and back were the
same: both had SPV security\footnote{This means using the \hashsig not only for determining the order of transactions, but also their validity.
In other words, this means trusting miners to not create invalid blocks.}.

An alternate scheme is an \emph{asymmetric} two-way peg: here users of the
\sidechain are full validators of the \mainchain, and transfers from
\mainchain to \sidechain do not require SPV proofs, since all validators
are aware of the state of the \mainchain. On the other hand, the
\mainchain is still unaware of the \sidechain, so SPV proofs are required
to transfer back.

This gives a boost in security, since now even a 51\% attacker cannot
falsely move coins from the \mainchain to the \sidechain. However, it
comes at the expense of forcing \sidechain validators to track the
\mainchain, and also implies that reorganisations on the \mainchain
may cause reorganisations on the \sidechain. We do not explore this possibility in detail here, as issues surrounding reorganisations result in a significant expansion in complexity.

\section{Drawbacks}

While \sidechains provide solutions to many problems in the cryptocurrency
space, and create countless opportunities for innovation to Bitcoin, they
are not without their drawbacks. In this section we review a few potential
problems, along with solutions or workarounds.

\subsection{Complexity}

Sidechains introduce additional complexity on several levels.

On the network level, we have
multiple independent unsynchronised blockchains supporting
transfers between each other. They must support transaction scripts
which can be invalidated by a later reorganisation proof. We
also need software which can automatically detect misbehaviour and produce
and publish such proofs.

On the level of assets, we no longer have a simple ``one chain, one asset''
maxim; individual chains may support arbitrarily many assets, even ones
that did not exist when the chain was first created. Each of these assets
is labelled with the chain it was transferred from to ensure
that their transfers can be unwound correctly.

Enabling the blockchain infrastructure to handle advanced features isn't sufficient: user interfaces for managing wallets will need to be reconsidered.
Currently in the \altcoin world, each chain has its own wallet which supports
transactions of that chain's coin. These will need to adapt to support
multiple chains (with potentially different feature sets) and transfers of assets
between chains. Of course, there is always the option of not using some functionality when the interface would be too complex.

\subsection{Fraudulent transfers\label{fraud}}

Reorganisations of arbitrary depth are in principle possible, which could allow
an attacker to completely transfer coins between \sidechains before causing
a reorganisation longer than the contest period on the sending chain to undo its half of the transfer. The result
would be an imbalance between the number of coins on the recipient chain and
the amount of locked output value backing them on the sending chain. If the
attacker is allowed to return the transferred coins to the original chain,
he would increase the number of coins in his possession at the expense
of other users of the sidechain.

Before discussing how to handle this, we observe that this risk can be made
arbitrarily small by simply increasing the contest period for transfers. Better,
the duration of the contest period could be made a function of the relative hashpower
of the two chains: the recipient chain might only unlock coins given an SPV proof
of one day's worth of \emph{its own} proof-of-work, which might correspond to
several days of the sending chain's proof-of-work. Security parameters like these
are properties of the particular sidechain and can be optimised for each sidechain's
application.

Regardless of how unlikely this event is, it is important that the \sidechain
not respond with catastrophic failure. It is possible to create an SPV proof
witnessing such an event, and \sidechains may accept such proofs. They may
be designed to react in one of many possible ways:
\begin{itemize}
\item No reaction. The result is that the \sidechain is a ``fractional reserve''
of the assets it is storing from other chains. This may be acceptable for tiny
amounts which are believed to be less than the number of lost \sidecoins,
or if an insurer promises to make good on missing assets. However,
beyond some threshold, a ``bank run'' of withdrawals from the \sidechain is likely,
which would leave somebody holding the bag in the end. Indirect damage could include
widespread loss of faith in \sidechains, and the expense to the \mainchain to process
a sudden rush of transactions.
\item The peg and all dependent transactions could be
reversed. However, as coins tend to diffuse and histories intermingle, the effect of such a
reversal could be catastrophic after even short periods of time. It also limits fungibility, as
recipients would prefer coins with ``clean'' histories (no recent pegs). We expect that such a loss of fungibility might have disastrous consequences.

\item The amount of all coins could be reduced,
while leaving the exchange rate intact. Now users who transferred coins to the sidechain
prior to the attack are disadvantaged relative to new ones. Reducing the exchange rate for
sidechain coins would be equivalent.
\end{itemize}

Many variations on these reactions are possible: for example, temporarily decreasing the exchange
rate so those who ``make a run'' on the \sidechain cover the loss of those who don't.

\subsection{Risk of centralisation of mining}

An important concern is whether the introduction of \sidechains with mining
fees places resource pressure on miners, creating Bitcoin centralisation risks.

Because miners receive compensation
from the block subsidy and fees of each chain they provide work
for, it is in their economic
interest to switch between providing \hashsigs for different similarly-valued
blockchains following changes in difficulty and movements in market value.

One response is that some blockchains have tweaked their blockheader definition such
that it includes a part of Bitcoin's \hashsig, thus enabling miners to provide a single
\hashsig that commits to Bitcoin as well as one or more other blockchains --- this is
called \emph{merged mining}.
Since merged mining enables re-use of work for multiple blockchains, miners are
able to claim compensation from each blockchain that they provide \hashsigs for.

As miners provide work for more blockchains, more resources are needed
to track and validate them all.
Miners that provide work for a subset of blockchains are compensated less than
those which provide work for every possible blockchain.
Smaller-scale miners may be unable to afford the full costs to mine
every blockchain, and could thus be put at a disadvantage compared to larger,
      established miners who are able to claim greater compensation from a larger
      set of blockchains.

      We note however that it is possible for miners to delegate validation and
      transaction selection of any subset of the blockchains that they provide work for.
      Choosing to delegate authority enables miners to avoid almost all of the
      additional resource requirements, or provide work for blockchains that they
      are still in the process of validating.
      However such delegation comes at the cost of centralising validation and
      transaction selection for the blockchain, even if the work generation itself
      remains distributed.
      Miners might also choose instead to not provide work for blockchains that they
      are still in the process of validating, thus voluntarily giving up some
      compensation in exchange for increased validation decentralisation.

\subsection{Risk of soft-fork\label{softfork}}
In Bitcoin, a \emph{soft-fork} is an addition to the Bitcoin protocol made
backwards compatible by being designed to strictly reduce the set of valid
transactions or blocks. A soft-fork can be implemented with merely a supermajority
of the mining computational power participating, rather than all full nodes.
However, participants' security with respect to the soft-forked features is only SPV-level until they upgrade.
Soft-forks have been used several times to deploy new features
and fix security issues in Bitcoin (see \cite{andresen2012-1}).

A two-way peg, implemented as described in this paper, has only SPV security
and therefore has greater short-term dependence on miner honesty than Bitcoin
does (see the attack described in Section \ref{fraud}). However, a two-way peg can be
boosted to security absolutely equal to Bitcoin's if all full nodes on both
systems inspect each other's chain and demand mutual validity as a
soft-forking rule.

A negative consequence of this would be loss of isolation of any
soft-fork-required sidechain. Since isolation was one of the goals of using
\peggedsidechains, this result would be undesirable unless a sidechain was almost
universally used. Absent \peggedsidechains, however, the next alternative
would be to deploy individual changes as hard- or soft-forks in Bitcoin
directly. This is even more abrupt, and provides no real mechanism for the
new facility to prove its maturity and demand before risking Bitcoin's
consensus on it.

\section{Applications\label{app}}

With the technical underpinnings out of the way, in this section we explore user-facing
applications of \sidechains, which effectively extend Bitcoin to do things
that it cannot today.

\subsection{Altchain experiments}
The first application, already mentioned many times, is simply creating \altchains with coins
that derive their scarcity and supply from Bitcoin. By using a \sidechain which carries
bitcoins rather than a completely new currency, one can avoid the thorny problems of initial distribution and market vulnerability, as well as barriers to adoption for new users, who no
longer need to locate a trustworthy marketplace or invest in mining hardware to obtain \altcoin assets.

\subsubsection{Technical experimentation}
Because \sidechains are technically still fully-independent chains, they
are able to change features of Bitcoin such as block structure or                              
transaction chaining. Some examples of such features are:                                      
\begin{itemize} 
\item By fixing undesired transaction malleability\footnote{Note that some forms of malleability are desired (\emph{i.e.} the types provided by SIGHASH flags other than SIGHASH\_ALL).} --- which can only be fixed partially in Bitcoin \cite{wuille2014} --- protocols which
involve chains of unconfirmed transactions can be executed safely. Transaction                 
malleability is a problem in Bitcoin which allows arbitrary users to tweak
transaction data in a way that breaks any later transactions which depend on
them, even though the actual content of the transaction is unchanged. An example               
of a protocol broken by transaction malleability is \emph{probabilistic                        
payments}\cite{caldwell2012}.                                                                  

\item Improved payer privacy, \emph{e.g.} the ring signature scheme used by Monero, can reduce the systemic risk of the
transactions of particular parties being censored, protecting the fungibility of the
cryptocurrency. Improvements to this have been suggested by Maxwell and Poelstra               
\cite{maxwell+poelstra2014, poelstra2014-2} and Back\cite{back2013-2}, which would allow for even greater privacy.
Today, ring signatures can be used with Monero coins, but not bitcoins; \sidechains would
avoid this exclusivity.

\item Script extensions (for example, to efficiently support coloured                           
coins\cite{jl20122013}) have been proposed for Bitcoin. Since such
extensions are usable only by a small subset of users, but all users                           
would need to deal with the increased complexity and risk of subtle                            
interactions, these extensions have not been accepted into Bitcoin.                            

Other suggested script extensions include support for new cryptographic                        
primitives. For example, Lamport signatures\cite{lamport1979}, while                           
large, are secure against quantum computers.                                                   

\item Many ideas for extending Bitcoin in incompatible ways are described                      
at \cite{maxwell2014} and at \url{http://www.bitcoin.ninja}.
\end{itemize}                                                                                  

Since changes like these affect only the transfer of coins,                        
rather than their creation, there is no need for them to require a separate currency. With \sidechains, users can safely and                            
temporarily experiment with them. This encourages adoption for the
\sidechain, and is less risky for participants relative to using an entirely separate \altcoin.                                                          
                                                                                               
\subsubsection{Economic experimentation}

Bitcoin's reward structure assigns new coins to miners. 
This effectively inflates the currency but it winds down over time according to a
step-wise schedule. Using this inflation to subsidise mining has been a successful
complement to transaction fees to secure the network.

An alternate mechanism for achieving block rewards on the \sidechain is \href{http://en.wikipedia.org/wiki/Demurrage\_(currency)}{demurrage},
an idea pioneered for digital currency by Freicoin (\url{http://freico.in}).
In a demurring cryptocurrency, all unspent outputs lose value over time, with the lost
value being recollected by miners. This keeps the currency supply stable while
still rewarding miners. It may be better aligned with user interests than inflation because loss to demurrage is
enacted uniformly everywhere and instantaneously, unlike inflation; it also
mitigates the possibility of long-unspent ``lost'' coins being reanimated at their
current valuation and shocking the economy, which is a perceived risk in Bitcoin.
Demurrage creates incentives to increase monetary velocity and lower interest
rates, which are considered (\emph{e.g.} by Freicoin advocates and other supporters of
Silvio Gesell’s theory of interest\cite{gesell1916}) to be socially
beneficial. In \peggedsidechains, demurrage allows miners to
be paid in existing already-valued currency.

Other economic changes include required miner fees, transaction reversibility,
outputs which are simply deleted once they reach a certain age, or
inflation/demurrage rates pegged to events outside of the \sidechain. All of these changes
are difficult to do safely, but the ease of creation and reduced risk of \sidechains
provide the necessary environment for them to be viable.

\subsection{Issued assets}

To this point, we have mostly been thinking about \sidechains which do not need
their own native currency: all coins on the \sidechain are initially locked, until
they are activated by a transfer from some other \sidechain. However, it is
possible for \sidechains to produce their own tokens, or \emph{issued assets},
which carry their own semantics. These can be transferred to other \sidechains
and traded for other assets and currencies, all without trusting a central party, even if
a trusted party is needed for future redemption.

Issued asset chains have many applications, including traditional financial instruments such
as shares, bonds, vouchers, and IOUs. This allows external protocols to delegate ownership
and transfer tracking to the \sidechain on which the ownership shares were issued. Issued
asset chains may also support more innovative instruments such as smart property.

These technologies can also be used in \emph{complementary currencies}\cite{lietaer2001}.
Examples of complementary currencies include community currencies, which are
designed to preferentially boost local businesses; business barter associations, which
support social programs like education or elderly care; and limited-purpose tokens which
are used within organisations such as massive multiplayer games, loyalty programs, and
online communities\footnote{For more information on complementary currencies, see
\url{http://www.complementarycurrency.org/}}.

 A suitably extended scripting system and an asset-aware transaction format would allow the
creation of useful transactions from well-audited components, such as the merger
of a bid and an ask to form an exchange transaction,  enabling the
creation of completely trustless peer-to-peer marketplaces for asset
exchange and more complex contracts such as trustless
options\cite{friedenbach+timon2013}. These contracts could, for example, 
help reduce the volatility of bitcoin itself.

\section{Future directions}

\subsection{Hashpower attack resistance}

The main thrust of this paper surrounds \twowp using SPV proofs, which are
forgeable by a 51\%-majority and blockable by however much hashpower is needed
to build a sufficiently-long proof during the transfer's contest period. (There is a tradeoff
on this latter point --- if 33\% hashpower can block a proof, then
67\% is needed to successfully use a false one, and so on.)

Some other ideas worth exploring in \sidechains are:
\begin{itemize}
\item \textbf{Assurance contracts.} The sidechain's transaction fees are withheld
from miners unless their hashpower is at least, say, 66\% of that of Bitcoin.
These sorts of contracts are easy for a cryptocurrency to implement, if they are
designed in from the start, and serve to increase the cost of blocking transfers.

\item \textbf{Time-shifted fees.} Miners receive part of their fees in a block
far in the future (or spread across many blocks) so that they have incentive to
keep the chain operational.

This may incentivise miners to simply receive fees out-of-band, avoiding the need
to wait for future in-chain rewards. A variation on this scheme is for miners to receive
a token enabling them to mine a low-difficulty block far in the future; this has the same
effect, but directly incentivises its recipient to mine the chain.

\item \textbf{Demurrage.} Block subsidies can be given to miners through
demurrage to incentivise honest mining. Since only as
much can be transferred to Bitcoin or another \sidechain as was transferred out,
this fund reallocation would be localised to the \sidechain in which it occurs.

\item \textbf{Subsidy.} A sidechain could also issue its own separate native currency as reward,
effectively forming an \altcoin. However, these coins would have a free-floating value and as a
result would not solve the volatility and market fragmentation issues with \altcoins.

\item \textbf{Co-signed SPV proofs.} Introducing signers who must sign
off on valid SPV proofs, watching for false proofs. This results in a  direct tradeoff between centralisation and security against a high-hashpower attack. There is a wide spectrum of trade-offs available in this area: signers may be required only for high-value transfers;
they may be required only when \sidechain hashpower is too small a percentage of Bitcoin's; etc. Further discussion about the usefulness of this kind of trade-off is covered in Appendix \ref{padaptor}.

\item \textbf{SNARKs.} An exciting recent development in academic cryptography has been the invention of
SNARKs \cite{ben-sasson+chiesa+genkin+tromer+virza2013}. SNARKs are space-efficient,
quickly verifiable zero-knowledge cryptographic proofs that some computation was done.
However, their use is currently inhibited because the proofs for most programs are too slow to generate on today's
computers, and the existing constructions require a \emph{trusted setup}, meaning that the creator of the system
is able to create false proofs.

A futuristic idea for a low-value or experimental \sidechain is to invoke a trusted authority,
whose only job is to execute a trusted setup for a SNARK scheme. Then blocks could be constructed
which prove their changes to the unspent-output set, but do so in zero-knowledge in the
actual transactions. They could even commit to the full verification of all previous
blocks, allowing new users to get up to speed by verifying only the single latest block.
These proofs could also replace the \hashsigs used to move coins from another chain by
proving that the sending chain is valid according to some rules previously defined.
\end{itemize}

\section{Acknowledgements}

We would like to thank Gavin Andresen, Corinne Dashjr, Mathias Dybvik, Daniel Folkinshteyn, Ian Grigg, Shaul Kfir,
midnightmagic, Patrick Strateman, Kat Walsh, and Glenn Willen for reviewer
comments.

\begin{appendices}
\section{Federated peg\label{padaptor}}

One of the challenges in deploying pegged sidechains is that Bitcoin script
is currently not expressive enough to encode the verification rules for an SPV
proof.
The required expressiveness could be added in a safe, compatible, and highly
compartmentalised way (\emph{e.g.}, by converting a no-op instruction into an
\texttt{OP\_SIDECHAINPROOFVERIFY} in a soft-fork). However, the
difficulty of building consensus for and deploying even simple new features
is non-trivial. Recall these difficulties were part of the motivation for
pegged sidechains to begin with. What we want is a way to try out future script capabilities for Bitcoin without deploying them everywhere.

Fortunately, by adopting some additional security assumptions at the expense of the low trust design objective, it is possible
to do an initial deployment in a completely permissionless way. The key
observation is that any enhancement to Bitcoin Script can be implemented
externally by having a trusted federation of mutually distrusting
\emph{functionaries}\footnote{
From Wiktionary (\url{https://en.wiktionary.org/wiki/functionary}),
a functionary is
\begin{quote}
A person \ldots who holds limited authority and primarily serves to carry
out a simple function for which discretion is not required.
\end{quote}
We use this term to emphasise that while functionaries have the physical power
to disrupt transfers between \sidechains, their correct operation is purely
mechanical.
} evaluate the script and accept by signing for an ordinary multisignature script. That is, the functionaries act as a \emph{protocol
adaptor} by evaluating the same rules we would have wanted Bitcoin to evaluate,
but cannot for lack of script enhancements. Using this we can achieve a
\emph{federated peg}.

This approach is very similar to the approach of creating a multi-signature
off-chain transaction system, but the required server-to-server consensus
process is provided by simply observing the blockchains in question.
The result is a deterministic, highly-auditable process which simplifies the
selection and supervision of functionaries. Because of these similarities, many
of the techniques used to improve security and confidence in off-chain
payment systems can be employed for federated pegs. For example:
functionaries can be geographically diverse, bonded via escrowed coins or
expensive-to-create coercion-resistant pseudonymous identities, implemented on remote-attesting tamper-resistant hardware, and so on\cite{Todd2013}.
For small-scale uses, owners of coins in the system can themselves act as the functionaries, thus avoiding third party trust.

Once sidechains with a federated peg are in use, the addition of SPV
verification to Bitcoin script can be seen as merely a security upgrade to reduce
the trust required in the system. Existing sidechains could simply migrate
their coins to the new verification system. This approach also opens
additional security options: the \hashsig
provided by mining is not very secure for small systems, while the trust
of the federation is riskier for large systems. A sidechain could adaptively
use both of these approaches in parallel, or even switch based on apparent
hashrate.

Consider the example of a sidechain using a 3 of 5 federation of
functionaries to implement a two-way peg with Bitcoin. The federation
has secp256k1 public points (public keys) $P_1$, $P_2$, $P_3$, $P_4$, and $P_5$ and a
redeemscript template \texttt{3 x x x x x 5 OP\_CHECKMULTISIG}
known to all participants in the sidechain. To send coins to a ScriptPubKey $SPK$,
a user who wants the coins to become available on a sidechain using the federated peg computes a cross-chain P2SH\cite{andresen2012-2} address by the following key derivation scheme:

% Check no page break here!

\begin{algorithm}
\caption{GenerateCrossChainAddress}
\begin{algorithmic}[1]
\Require{A target ScriptPubKey $SPK$ which will receive the coins in the other chain}
\Require{A list $\{P_i\}_{i=1}^n$ of the functionaries' public points}
\Require{A redeemScript $template$ describing the functionary requirements}
\Ensure{A P2SH address}
\Ensure{Nonce used for this instance}

  \State $nonce \gets \texttt{random\_128bit}()$
  \For{$i \gets [1, n]$}
    \State $Tweak_i \gets \texttt{HMAC-SHA256}(key=P_i,~data=nonce || SPK)$
    \If{$Tweak_i >= \texttt{secp256k1\_order}$}
      \State Go back to start.
    \EndIf
    \State $PCC_i = P_i + G \times Tweak_i$
  \EndFor
  \State $address \gets \texttt{P2SH\_Multisig}(template, keys=PCC))$
\end{algorithmic}
\end{algorithm}
This derivation scheme is based on the same homomorphic technique \cite{maxwell2011}  used in BIP32 to allow third parties to derive publicly unlinkable addresses. It is the same underlying construction as a pay-to-contract transaction \cite{gerhardt+hanke2012}.
After generating the address, coins can be paid to it, and the user can later receive the resulting coins on the \sidechain by providing
the functionaries with the nonce, ScriptPubKey, and an SPV proof to help
them locate the payment in the blockchain. In order to aid third-party verification of the sidechain, these values could be included in the sidechain itself. Because the transfer is made by paying to a standard P2SH address and can pay to any ScriptPubKey, all Bitcoin services which can pay to a multisignature address will immediately be able to pay into, or receive payments from, a user using a federated sidechain.

The federated peg approach necessarily compromises on trust, but requires no changes
to Bitcoin --- only the participants need to agree to use it and only the participants
take the costs or risks of using it. Further, if someone wanted to prevent other
people from using a sidechain they could not do so: if the federated peg is used
privately in a closed community, its use can be made undetectable and uncensorable.
This approach allows rapid deployment and experimentation and will allow the
community to gain confidence in pegged sidechains before adopting any changes to the Bitcoin protocol.


\section{Efficient SPV proofs\label{compactspv}}

In order to transfer coins from a \sidechain back to Bitcoin, we need to
embed proofs that \sidechain coins were locked in the Bitcoin blockchain.
These proofs should contain (a) a record that an output was created in the
\sidechain, and (b) a \hashsig proving sufficient work on top of this output.
Because Bitcoin's blockchain is shared and validated by all of its participants,
these proofs must not impose much burden on the network.
Outputs can be easily recorded compactly,
but it is not obvious that the \hashsig can be.

\paragraph{Compact SPV Security.} The confidence in an SPV proof can be justified
by modelling an attacker and the honest network as random processes~\cite{miller+laviola2014}.
These random processes have a useful statistical property: while each hash
must be less than its target value to be valid, half the time it will be less than half the target;
a third of the time it will be less than a third the target; a quarter of the time less than
a quarter the target; and so on. While the hash value itself does not change the amount of work
a block is counted as, the presence of lower-than-necessary hashes is in fact statistical evidence
of more work done in the chain\cite{miller2012}.
We can exploit this fact to prove equal amounts of work with only a few block
headers\cite{friedenbach2014}. It should therefore be possible to greatly
compress a list of headers while still proving the same amount of work. We
refer to such a compressed list as a \emph{compact SPV proof} or
\emph{compressed \hashsig}.

However, while the expected work required to produce a fraudulent
compact SPV proof is the same as that for a non-compact one, a forger's probability
of success no longer decays exponentially with the amount of work proven:
a weak opportunistic attacker has a much higher probability of
succeeding ``by chance’’; \emph{i.e.}, by finding low hashes early. To illustrate
this, suppose such an attacker has 10\% of the network’s hashrate, and is trying
to create an SPV proof of 1000 blocks before the network has produced this many.
Following the formula in \cite{nakamoto2009} we see that his likelihood of success is
\[ 1 - \sum_{k=0}^{1000} \frac{100^k e^{-100}}{k!} (1 - (0.1)^{1000-k}) \approx 10^{-196} \]
To contrast, the same attacker in the same time can produce a single block
proving 1000 blocks' worth of work with probability roughly 10\%, a much
higher number.

A detailed analysis of this problem and its possible solutions is out of scope
for this document. For now we will describe an implementation of compact SPV
proofs, along with some potential solutions to block this sort of attack while
still obtaining significant proof compaction.

Note that we are assuming a constant difficulty. We observe that Bitcoin's difficulty,
while non-constant, changes slowly enough to be resistant to known
attacks\cite{bahack2013}. We therefore expect that corrections which take into
account the adjusting difficulty can be made.
\paragraph{Implementation.} The inspiration for compact SPV proofs is the
\emph{skiplist} \cite{pugh1990}, a probabilistic data structure which provides
log-complexity search without requiring rebalancing (which is good because an
append-only structure such as a blockchain cannot be rebalanced).

We require a change to Bitcoin so that rather than each blockheader committing
only to the header before it,
it commits to every one of its ancestors. These commitments can be stored
in a Merkle tree for space efficiency: by including only a root hash in
each block, we obtain a commitment to every element in the tree. Second,
when extracting SPV proofs,
provers are allowed to use these commitments to jump back to a block more
than one link back in the chain, \emph{provided} the work actually proven
by the header exceeds the total target work proven by only following direct
predecessor links. The result is a short \hashsig which proves just as much
work as the original blockchain.

How much smaller is this? Suppose we are trying to produce an SPV proof of
an entire blockchain of height $N$. Assume for simplicity that difficulty
is constant for the chain; \emph{i.e.}, every block target is the same.
Consider the probability of finding a large enough proof to skip all the
way back to the genesis within $x$ blocks; that is, between block $N - x$
and block $N$. This is one minus the probability we \emph{don't}, or
\[ 1 - \prod_{i=1}^x \frac{N - i}{N - i + 1} = 1 - \frac{N - x}{N}  = \frac{x}{N}\]

The expected number of blocks needed to scan back before skipping the
remainder of the chain is thus
\[ \sum_{x = 1}^N \frac{x}N = \frac{N + 1}2 \]

Therefore if we want to skip the entire remaining chain in one jump, we expect
to search only halfway; by the same argument we expect to skip this half after
only a quarter, this quarter after only an eighth, and so on. The result is
that the expected total proof length is logarithmic in the original length of
the chain.

For a million-block chain, the expected proof size for the entire chain is
only $\log_2 1000000 \approx 20$ headers. This brings the \hashsig size down into the tens-of-kilobytes
range.


However, as observed above, if an attacker is able to produce compact proofs in
which only the revealed headers are actually mined, he is able to do so with
non-negligible probability in the total work being proven. One such strategy
is for the attacker to produce invalid blocks in which every backlink
points to the most recent block. Then when extracting a compact proof, the
attacker simply follows the highest-weighted link every time.

We can adapt our scheme to prevent this in one of several ways:

\begin{itemize}
\item By limiting the maximum skip size, we return to Bitcoin's property that
the likelihood of a probabilistic attack decays exponentially with the amount
of work being proven. The expected proof size is smaller than a full list of
headers by a constant (proportional to the maximum skip size) factor.

\item By using a maximum skip size which increases with the amount of work
being proven it is possible to get sublinear proof sizes, at the cost of
subexponential decay in the probability of attack success. This gives greater
space savings while still forcing a probabilistic attacker's likelihood of
success low enough to be considered negligible.

\item Interactive approaches or a cut-and-choose mechanism may allow
compact proofs with only a small security reduction. For example, provers
might be required to reveal random committed blockheaders (and their connection to
the chain), using some part of the proof as a random seed. This reduces the
probability of attack while only increasing proof size by a constant factor.
\end{itemize}

If we expect many transfers per \sidechain, we can maintain a special
output in the \mainchain which tracks the \sidechain's tip. This output is
moved by separate SPV proofs (which may be compacted in one of the above
ways), with the result that the \mainchain is aware of a recent \sidechain's
tip at all times.

Then transfer proofs would be required to always end at this tip, which can
be verified with only a single output lookup. This
guarantees verifiers that there are no ``missing links'' in the transfer
proofs, so they may be logarithmic in size without increased risk of forgery.

This makes the total cost to the \mainchain proportional to the number of
\sidechains and their length; without this output, the total cost is also
proportional to the number of inter-chain transfers.

This discussion is not exhaustive; optimising these tradeoffs and formalising the
security guarantees is out of scope for this paper and the topic of ongoing
work.

\section{Atomic swaps\label{atomicswaps}}

Once a \sidechain is operational, it is possible for users to exchange coins
atomically between chains, without using the peg. In fact, this is possible
with \altcoins today, though the independent prices make it harder to organise.
This is important, because as we have seen, direct use of the peg requires
fairly large transactions (with correspondingly large fees) and long wait periods. To contrast,
atomic swaps can be done using only two transactions on each network, each of
size similar to ordinary pay-to-address transactions.

One such scheme, due to Tier Nolan\cite{nolan2013}, works as follows.

Suppose we have two parties, $A$ and $B$, who hold coins on different blockchains.
Suppose also that they each have addresses $\pk_A$ and $\pk_B$ on the other's
chain, and that $A$ has a secret number $a$. Then $A$ can
exchange coins for $B$'s as follows:
\begin{enumerate}
\item On one chain, $A$ creates a transaction moving coins to an output $O_1$
which can only be redeemed with (a) a revealing of $a$ and $B$'s signature, or (b)
both $A$ and $B$'s signatures. $A$ does not yet broadcast this.

$A$ creates a second transaction returning the coins from $O_1$ to $A$, with
a locktime\footnote{In Bitcoin, a transaction's \emph{locktime} prevents it from
being included in the blockchain until some timeout has expired. This is useful
for creating refunds in interactive protocols which can only be redeemed if the
protocol times out.} of 48 hours. $A$ passes this transaction to $B$ to be
signed.

Once $B$ signs the locked refund transaction, $A$ may safely broadcast the
transaction moving coins to $O_1$, and does so.

\item Similarly, $B$ creates a transaction moving coins to an output $O_2$
on the other chain, which can only be redeemed by (a) a revealing of $a$ and $A$'s
signature, or (b) both $A$ and $B$'s signatures. $B$ does not yet broadcast this.

$B$ creates a second transaction returning the coins from $O_2$ to $B$, with
a locktime of 24 hours. $B$ passes this transaction to $A$ to be signed.

Once $A$ signs the locked refund transaction, $B$ may safely broadcast the
transaction moving his coins to $O_2$, and does so.

\item Since $A$ knows $a$, $A$ is able to spend the coins in $O_2$, and does
so, taking possession of $B$'s coins.

As soon as $A$ does so, $a$ is revealed and $B$ becomes able to spend the
coins in $O_1$, and does so, taking possession of $A$'s coins.
\end{enumerate}

\end{appendices}

\nolinenumbers

\clearpage
\bibliographystyle{amsalpha}
\bibliography{asp}

\end{document}

